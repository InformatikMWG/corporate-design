% (c) 2018 Sebastian Friedl <sfr682k@t-online.de>
% This document gets drafted and will be finished soon.
%
% This work currently consisting of the file styleguide.tex
% is subject to the LaTeX Project Public License, Version 1.3c or - at your option - any later version.
%
% The work has the LPPL status 'maintained'
% Current maintainer of the work is Sebastian Friedl

% !TeX spellcheck=en_US


\documentclass[10pt,DIV=11,abstracton]{scrartcl} 


\usepackage{iftex}
\RequireLuaTeX

\usepackage{fontspec}
\setmainfont{FiraSans}
\setsansfont{FiraSans}
\setmonofont{FiraMono}

\usepackage{polyglossia}
\setdefaultlanguage{english}

\usepackage[english]{selnolig}



\usepackage[pdfborder={0 0 0}]{hyperref}

\usepackage{array}
\usepackage{booktabs}
\usepackage{multirow}

\usepackage{verbatim}

\usepackage{xcolor}
\definecolor{MWGRed}{RGB}{100,29,27}

\linespread{1.05}

\def\informatikMWG{\texttt{InformatikMWG}}


\parindent0pt

\usepackage{lipsum}

\title{\informatikMWG\ Style Guide}
\subtitle{for publications and presentations}
\author{Sebastian Friedl}
\date{Draft: \today}

\hypersetup{pdftitle={Draft: InformatikMWG Style Guide}, pdfauthor={Sebastian Friedl}}

\begin{document}
    \maketitle

    \begin{abstract}\noindent
        This document specifies the layout and appearance of official documents published by members of the \informatikMWG\ organization.
    \end{abstract}
    
    \bigskip\bigskip\bigskip
    
    \section{Official Templates}
    We aim to provide official \informatikMWG\ \LaTeX\ document classes following this style guide. Currently, work is in progress; however, it may take some long time until they are finished and matured.

    \medskip
    We don't --- and we won't --- provide official templates for M\$ Office. On purpose.

    
    \section{General Style specifications}
    This section describes general style specifications to be used for presentations and (smaller) publications.
    Please note that these specifications do \emph{\textbf{not}} apply to project documentations.
    
    \subsection{Page layout}
    \subsubsection*{Presentations}
    Use frame dimensions of 16\,cm\,$\times$\,9\,cm and the default margin sizes.
    
    \subsubsection*{Articles and Reports}
    Construct the page layout by dividing the page in 11 equal-sized rows and columns (subtract the binding correction from the page width). The margin sizes are depicted in table~\ref{tab:margins} with \texttt{rh} representing a row height and \texttt{cw} representing a column width.
    
    \begin{table}\centering
        \begin{tabular}{l@{\quad}>{\ttfamily}r@{\qquad\quad}*{2}{>{\ttfamily}r<{\,cm}}@{\qquad}*{2}{>{\ttfamily}r<{\,cm}}}\toprule
        	&& \multicolumn{2}{c}{portrait} & \multicolumn{2}{c}{landscape} \\
            && \multicolumn{1}{c}{A4} & \multicolumn{1}{c}{A5} & \multicolumn{1}{c}{A4} & \multicolumn{1}{c}{A5} \\\midrule
            %
        	top margin    &   1\,rh & 2.70 & 1.91 & 1.91 & 1.35 \\
        	bottom margin &   2\,rh & 5.40 & 3.82 & 3.82 & 2.69 \\\cmidrule{1-6}
        	left margin   & 1.5\,cw & 2.86 & 2.02 & 4.05 & 2.86 \\
        	right margin  & 1.5\,cw & 2.86 & 2.02 & 4.05 & 2.86 \\\cmidrule{1-6}
        	inner margin  &   1\,cw & 1.91 & 1.35 & 2.70 & 1.91 \\
        	outer margin  &   2\,cw & 3.82 & 2.69 & 5.40 & 3.82 \\\bottomrule
        \end{tabular}
    
        \caption{Margin sizes for some common paper formats}
        \label{tab:margins}
    \end{table}
    
    \medskip
    \LaTeX\ users may rely on the \verb|typearea| package. \\
    The “correct” way to load it depends on the used document class:
    
    \begin{itemize}
        \item Standard \LaTeX\ document classes (\verb|article|, \verb|report|, \verb|book|): \\
            Add \verb|\usepackage[DIV=11]{typearea}| to your preamble
            
        \item KOMA Script document classes (\verb|scrartcl|, \verb|scrreprt|, \verb|scrbook|): \\
            Add the \verb|DIV=11| option to the \verb|\documentclass| command
    \end{itemize}
    
    
    \subsection{Fonts and their usage}
    The corporate design uses these two fonts:
    \begin{itemize}\itemsep0pt
        \item Fira Sans\quad for nearly everything
        \item \texttt{Fira Mono}\quad for code, user names, passwords, URLs and similar stuff
    \end{itemize}
    
    \medskip
    Both fonts are licensed under the OFL 1.1 and can be downloaded at:
    \begin{center}
        \url{http://bboxtype.com/typefaces/FiraSans} or \\
        \url{https://github.com/bBoxType/FiraSans}
    \end{center}
    
    \medskip
    \textbf{The following basic rules apply:} \nopagebreak
    \begin{enumerate}
        \item Whenever possible a font size of 10pt and line spread of 105\,\% should be used.
        \item Use lining figures (1234567890) instead of old style ones {(\fontspec[Numbers=OldStyle]{FiraSans}1234567890)}
        \item For part and section headings, the \textsc{Small Caps} shape should be used.
    \end{enumerate}
    
    
    \subsection{Colors}
    \begin{itemize}\itemsep0pt
        \item The main color used for text is \textbf{black}
        \item For structural elements and backgrounds, \textbf{\color{MWGRed}\texttt{MWGRed}} (\texttt{rgb(100,29,27)}) should be used
        \item If {\color{MWGRed}\texttt{MWGRed}} is used as background color, the foreground color should be white.
    \end{itemize}
    
    
\end{document}
