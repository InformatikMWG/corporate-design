% (c) 2018 Sebastian Friedl <sfr682k@t-online.de>
% This document gets drafted and will be finished soon.
%
% This work currently consisting of the file styleguide.tex
% is subject to the LaTeX Project Public License, Version 1.3c or - at your option - any later version.
%
% The work has the LPPL status 'maintained'
% Current maintainer of the work is Sebastian Friedl


\documentclass[abstracton]{scrartcl} 


\usepackage{iftex}
\RequireLuaTeX

\usepackage{fontspec}
\setmainfont{FiraSans}
\setsansfont{FiraSans}
\setmonofont{FiraMono}

\usepackage[pdfborder={0 0 0}]{hyperref}

\usepackage{verbatim}

\usepackage{xcolor}
\definecolor{MWGRed}{RGB}{100,29,27}


\def\informatikMWG{\texttt{InformatikMWG}}


\parindent0pt


\title{\informatikMWG\ Style Guide}
\subtitle{for articles and presentations}
\author{Sebastian Friedl}
\date{Draft: \today}

\hypersetup{pdftitle={Draft: InformatikMWG Style Guide}, pdfauthor={Sebastian Friedl}}

\begin{document}
    \maketitle

    \begin{abstract}
        This document specifies the layout and appearance of official documents published by members of the \informatikMWG\ organization.
    \end{abstract}
    
    \bigskip\bigskip\bigskip
    
    \section{Official Templates}
    We aim to provide official \informatikMWG\ \LaTeX\ document classes following this style guide. Currently, work is in progress; however, it may take some long time until they are finished and matured.

    \medskip
    We don't --- and we won't --- provide official templates for M\$ Office. On purpose.
    
    
    \section{General Style specifications}
    This section describes general style specifications to be used for presentations and (smaller) publications.
    Please note that these specifications do \emph{\textbf{not}} apply to project documentations.
    
    \subsection{Page layout}
    \textit{Presentations:} \\
    Use frame dimensions of 16\,cm\,$\times$\,9\,cm
    
    \medskip
    \textit{Articles and Reports:} \\
    Construct the page layout by dividing the page in 10 equal-sized rows and columns (subtract the binding correction from the page width). \\[\smallskipamount]
    Now, one is able to calculate \dots
    \begin{itemize}\itemsep0pt
        \item top margin size (the height of a row),
        \item bottom margin size (twice the height of a row),
        \item inner margin size (the width of a column) and
        \item outer margin size (twice the width of a column)
    \end{itemize}

    An even simpler solution for \LaTeX-Users: \verb|\usepackage{typearea}|
    
    
    \subsection{Fonts and their usage}
    \begin{itemize}\itemsep0pt
        \item Fira Sans\quad for nearly everything
        \item \texttt{Fira Mono}\quad for code, user names, passwords, URLs and similar stuff
    \end{itemize}
    
    \medskip
    Both fonts are licensed under the OFL 1.1 and can be downloaded at:
    \begin{center}
        \url{http://bboxtype.com/typefaces/FiraSans} or \\
        \url{https://github.com/carrois/Fira}
    \end{center}
    
    \medskip
    \textbf{The following basic rules apply:} \nopagebreak
    \begin{enumerate}
        \item Whenever possible a font size of 11pt should be used.
            For use in \texttt{beamer} presentations, Fira should be scaled down to 85\,\% of its natural size.
        
        \item Use lining figures (1234567890) instead of old style ones (\fontspec[Numbers=OldStyle]{FiraSans}1234567890)
        
        \item For part and section headings, the \textsc{Small Caps} shape should be used.
    \end{enumerate}
    
    
    \subsection{Colors}
    \begin{itemize}
        \item The main color used for text is \textbf{black}
        \item For structural elements and backgrounds, \textbf{\color{MWGRed}\texttt{MWGRed}} should be used (\texttt{RGB} specification: \texttt{100/29/27})
        \item If {\color{MWGRed}\texttt{MWGRed}} is used as background color, the foreground color should be white.
    \end{itemize}
    
    
\end{document}
